\documentclass{beamer}
\usepackage{booktabs}
\usepackage{amsmath}
\usepackage{blkarray}
\usepackage{tikz}

\mode<presentation>
{
%  \usetheme{Malmoe}
\usetheme{default}
%\usecolortheme{seahorse}
  % or ...

 \setbeamercovered{transparent}
  % or whatever (possibly just delete it)
 \setbeamertemplate{footline}[default]
 \setbeamertemplate{navigation symbols}{\insertslidenavigationsymbol\insertframenavigationsymbol\insertdocnavigationsymbol}
}


\usepackage[english]{babel}
% or whatever

%\usepackage[latin1]{inputenc}
% or whatever

%\usepackage{times}
%\usepackage[T1,T5]{fontenc}
% Or whatever. Note that the encoding and the font should match. If T1
% does not look nice, try deleting the line with the fontenc.


\title{Some ideas for library design}

\begin{document}
\begin{frame}
  \titlepage
\end{frame}

\begin{frame}
  \frametitle{Matching pursuit}
  See
  \href{https://en.wikipedia.org/wiki/Matching\_pursuit}{\texttt{https://en.wikipedia.org/wiki/Matching\_pursuit}}
  for background. The idea here is to consider library design as a
  (sparse) signal approximation problem. For our purpose, a signal is
  simply a weighted (PageRank iterated) scaffold network. We can
  formulate library design as follows: Let $f$ denote the desired
  signal (e.g., scaffolds derived from a set of actives in one or more
  assays). For a given library $D = \{g_1, g_2, \ldots,\}$ of
  scaffolds, we seek to approximate $f$ as
  \[ \tilde{f} = \sum_n a_ng_n,\]
  where $a_n$ is the associated weight of scaffold $g_n$. Our goal is
  to select a set of molecules that minimize $\|f - \tilde{f}\|$. To
  calculate $\tilde{f}$, we use the greedy matching pursuit algorithm
  as given in the wikipedia reference.
\end{frame}

\begin{frame}
  \frametitle{Bayesian library}
  Given a set of assay annotations $A =\{a_1,a_2,\ldots,a_n\}$ defined
  over a collections of assays, we would like to identify which of the
  annotations are likely to apply to a set of unscreened molecules. We
  formulate this in the context of Bayesian as follows. Let $F =
  \{f_1, f_2, \ldots, f_k\}$ be the set of features (e.g., 
  structural keys) derived from a set of molecules $M$. The
  posteriori probability of $a_i$ for a new set of molecules $L$ is simply
  \[ p(a_i | L) = \frac{p(L|a_i)P(a_i)}{P(L)}\]
  or its na\"ive Bayes formulation
  \[ p(a_i|L) = \frac{\prod_kp(g_k|a_i)P(a_i)}{P(L)}\]
  where $g_k$ is the distribution of the feature $f_k$ over $L$. If
  $f_k$ is binary, then one possible way to encode $g_k$ is the
  Bernoulli distribution with mean $m$ estimated from $M$.
\end{frame}
\end{document}

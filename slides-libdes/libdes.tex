\documentclass{beamer}
\usepackage{booktabs}
\usepackage{amsmath}
%%\usepackage{blkarray}
\mode<presentation>
{
%  \usetheme{Malmoe}
\usetheme{default}
%\usecolortheme{seahorse}
  % or ...

 \setbeamercovered{transparent}
  % or whatever (possibly just delete it)
 \setbeamertemplate{footline}[default]
 \setbeamertemplate{navigation symbols}{\insertslidenavigationsymbol\insertframenavigationsymbol\insertdocnavigationsymbol}
}
\usepackage[english]{babel}

\title{What can your library do for you?}
\author{Rajarshi Guha, Dac-Trung Nguyen, Ajit Jadhav\\
NIH NCATS}

\begin{document}

\begin{frame}
  \titlepage
\end{frame}

\begin{frame}
  \frametitle{Library Design}
  \begin{itemize}
  \item Historical collections and assay data provide information on how a set of compounds has faired
  \item Use (dis)similarity and machine learning to construct new collections that show similar behavior
    \begin{itemize}
    \item Plus various constraints
    \end{itemize}
  \item If sufficiently annotated, compound behavior can be correlated to assay and biology characteristics
  \end{itemize}
\end{frame}

\begin{frame}
  \frametitle{Two Questions}
  How likely are compounds, associated with a  given annotation, identified as active?
  \vskip 2em
  Given a new set of compounds, what sets of assay conditions (as implied by the annotations) will they be active in? 
\end{frame}

\begin{frame}
  \frametitle{Prior Work}
  \begin{itemize}
  \item BAO annotated datasets
    \begin{itemize}
    \item \href{http://www.ncbi.nlm.nih.gov/pubmed/24441647}{de Souza et al, 2014}; \href{http://www.ncbi.nlm.nih.gov/pubmed/23155465}{Vempati et al, 2012}
    \end{itemize}
  \item Analyzing HTS datasets using BAO
    \begin{itemize}
    \item \href{http://www.ncbi.nlm.nih.gov/pubmed/25512330}{Zander-Balderud et al, 2015}; \href{http://www.ncbi.nlm.nih.gov/pubmed/21471461}{Sch\"{u}rer et al, 2011}
    \end{itemize}
  \item Semi-automated annotation of assay descriptions using the BAO 
    \begin{itemize}
    \item \href{http://www.ncbi.nlm.nih.gov/pubmed/25165633}{Clark et al, 2014}
    \end{itemize}
  \end{itemize}
\end{frame}

\begin{frame}
  \frametitle{BAO 2.0}
  \includegraphics[height=3.0in]{bao}
\end{frame}

\begin{frame}
  \frametitle{Assay Modeling}
  \begin{center}
    \includegraphics[width=0.9\paperwidth]{assaymodel}    
  \end{center}
\end{frame}

\begin{frame}
  \frametitle{Workflow}
  \begin{itemize}
  \item Extract unique BAO terms and for each term identify annotated assays
  \item Extract active compounds from this set of assays
  \item Compute fingerprint bit distribution
  \item Use these conditional bit distributions to identify the BAO terms that describe the assay that they are likely to be active in
  \end{itemize}
\end{frame}

\begin{frame}
  \frametitle{Dataset Overview}
  \begin{itemize}
  \item Extracted 4010 Pubchem AIDs from BARD
  \item Primary, confirmation, counterscreening assays
  \item 154M outcomes
  \item 740K compounds
  \item 192 unique BAO terms
  \end{itemize}
\end{frame}

\begin{frame}
  \frametitle{Dataset Overview}
  \begin{center}
  \includegraphics[width=0.5\textwidth]{img-outcomehistogram}
  \includegraphics[width=0.5\textwidth]{img-bioassay-outcomes} \\
  \includegraphics[width=0.5\textwidth]{img-termdepth}
  \includegraphics[width=0.5\textwidth]{img-termassaycount} \\    
  \end{center}
\end{frame}

\begin{frame}
  \frametitle{Pitfalls}
  \centering{
    \parbox{0.75\textwidth}{
      \textit{\color{red}If sufficiently annotated, compound behavior can be 
        correlated to assay and biology characteristics}}
}
\vskip 1em
  \begin{itemize}
  \item A very abstract, possibly lossy, view of the effect of compounds on biology
  \item Depends on correct and meaningful annotations
  \item Annotations terms are context dependent, but this may not be considered when annotating a dataset
  \item Depends on well designed structural keys
  \end{itemize}
\end{frame}

\begin{frame}
  \frametitle{Acknowledgements}
  \begin{itemize}
  \item Qiong Cheng (U. Miami)
  \item Stephan Sch\"{u}rer (U. Miami)
  \end{itemize}
\end{frame}
\end{document}
